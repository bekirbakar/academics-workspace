\documentclass[conference, a4paper]{IEEEtran}

\IEEEoverridecommandlockouts

\usepackage{cite}
\usepackage[cmex10]{amsmath}
\usepackage{multirow}
\usepackage{array}
\usepackage[lofdepth,lotdepth]{subfig}
\usepackage[turkish]{babel}
\usepackage[utf8]{inputenc}
\usepackage[T1]{fontenc}

\ifCLASSINFOpdf
  \usepackage[pdftex]{graphicx}
\else
\fi

\hyphenation{op-tical net-works semi-conduc-tor}

\setlength{\textfloatsep}{5pt}

\AtBeginDocument{\renewcommand\tablename{TABLO}}
\AtBeginDocument{\renewcommand\abstractname{Abstract}}

\begin{document}

\IEEEpubid{\makebox[\columnwidth]{978-1-5386-1501-0/18/\$31.00 \copyright\ 2018 IEEE\hfill}
    \hspace{\columnsep}\makebox[\columnwidth]{}}

\title{Derin Sinir Ağları ile Tekrar Saldırılarının Tespiti\\
    Replay Spoofing Attack Detection Using Deep Neural Networks}

\author{\IEEEauthorblockN{Bekir Bakar ve Cemal Hanilçi}
    \IEEEauthorblockA{Elektrik-Elektronik Mühendisliği Bölümü\\
        Bursa Teknik Üniversitesi\\
        Bursa, Türkiye\\
        b.bakar@outlook.com, cemal.hanilci@btu.edu.tr}}

\maketitle

\begin{ozet}
    Son yıllarda konuşmacı doğrulama (KD) sistemlerine olan ilgi artmış ve KD sistemlerinin kullanımı yaygınlaşmıştır.
    Bu durum; yanıltma saldırılarının tespitini, gerçek sesin sahte sesten ayırt edilmesi, KD sistemleri için önemli
    bir araştırma konusuna dönüştürmüştür. Bu çalışmada KD sistemlerine daha önceden kaydedilmiş bir sesin tekrar
    oynatılması ile gerçekleştirilebilecek tekrar saldırılarının tespiti problemi ele alınmıştır. Mel frekansı kepstrum
    katsayıları (MFCC) ve uzun dönem ortalama spektrum (LTAS) istatistikleri özniteliklerinin, derin sinir ağları (DNN)
    sınıflandırıcısı ile tekrar saldırılarının tesit edilmesi gerçekleştirilmiştir. ASVspoof 2017 veritabanı ile
    yapılan deneylerde: DNN ile LTAS ve MFCC öznitelikleri sınıflandırıldığında, ASVspoof 2017 yarışmasının referans
    yöntemi olan sabit Q dönüşümü kepstral katsayılarının (CQCC) GMM sınıflandırıcısı ile modellenmesine göre daha iyi
    performans gösterdiği tespit edilmiştir.
\end{ozet}

\begin{IEEEanahtar}
    konuşmacı doğrulama, yanıltma saldırıları, saldırı tespiti, derin sinir ağları
\end{IEEEanahtar}

\begin{abstract}
    In recent years, there has been increased interest in speaker verification (SV) systems and their usage has become
    widespread. This situation made the detecting of spoofing attacks, the discrimination of genuine speech from
    spoofed speech, an important research area for speaker verification (SV) systems. In this study, detection of
    replay spoofing attacks where a pre-recorded speech signal is used to gain unauthorized access to ASV systems is
    studied. Mel frequency cepstral coefficients (MFCC) and long-term average spectrum (LTAS) statistics features are
    used to detect replay attacks using deep neural network (DNN) classifier. Experimental results using ASVspoof 2017
    database show that MFCC and LTAS features with DNN classifier outperforms the Gaussian mixture model (GMM)
    classifier with constant Q transform cepstral coefficients (CQCC) which is the baseline replay attack detection
    system of the ASVspoof 2017 challenge.
\end{abstract}

\begin{IEEEkeywords}
    speaker verification, spoofing attacks, anti-spoofing, deep neural networks
\end{IEEEkeywords}

\IEEEpeerreviewmaketitle

\IEEEpubidadjcol

\section{G{\footnotesize İ}r{\footnotesize İ}ş}
Konuşmacı doğrulama (KD)\cite{kinnunen2010overview}, verilen bir ses sinyalinin iddia edilen kişiye ait
olup olmadığının tespit edilmesini hedefleyen, kimlik kabul veya reddetme işlemidir. Tıpkı parmak izi tanıma, yüz
tanıma vb. biyometrik sistemlerde olduğu gibi; KD sistemleri kişisel verilerin güvenliği, çeşitli sistemlere erişim
kontrolü, banka işlemleri ve daha bir çok alanda kullanılmaktadır \cite{hadid2015biometrics}. Teknolojik gelişmeler ve
bu alanda yapılan çalışmaların artmasıyla birlikte, sistemlerin kullanım alanı gün geçtikçe artmaktadır.

KD ve diğer biyometrik sistemlerin kullanımının yagınlaşması, sistemlere yapılan yanıltma saldırılarını da beraberinde
getirmiştir. Yanıltma saldırısı, bir kişinin (saldırganın) erişim izni olmayan bir sisteme erişme girişimidir
\cite{hadid2015biometrics}. En son geliştirilen KD sistemleri de dahil olmak üzere, sistemlerin yanıltma saldırılarına
karşı savunmasız olduğu birçok çalışmada açıkca gösterilmiştir \cite{wu2015spoofing}.

Bir KD sistemine yapılması olası saldırı türleri; konuşmacının sesinin sentezlenerek oluşturulması
\emph{-ses sentezleme-}\cite{de2012evaluation}, çeşitli yazılımlar kulanılarak saldırganın herhangi bir sesi hedef
kişinin sesine dönüştürmesi \emph{-ses dönüştürme}-\cite{stylianou2009voice}, hedef konuşmacının sesinin taklit
edilmesi \emph{-taklit-}\cite{hautamaki2015automatic}, hedef konuşmacının önceden kaydedilmiş ses kaydının kullanılması
\emph{-tekrar oynatma-} \cite{wu2015spoofing} şeklinde alt ketegorilere ayrılabilir. Bu
saldırı türlerinden taklit saldırısı özel bir yetenek gerektirmesi nedeniyle, karşılaşılması güç bir saldırı türüdür
ve önemli bir tehdit olarak algılanmamaktadır. Ses sentezleme ve ses dönüştürme saldırıları kolay erişilebilir birçok
yazılım kullanılarak yapılabileceği ve saldırı başarı oranının yüksek olması nedeniyle önemli bir tehdit olarak
görülmektedir. Tekrar saldırıları ise; hemen herkesin sahip olduğu basit bir cep telefonu ile yapılabilmesi
sebebiyetiyle en önemli yanıltma saldırısı türüdür.

Literatürde bulunan bazı tekrar saldırısı tespit çalışmaları şöyle özetlenebilir: Villalba ve Lleida
\cite{villalba2010speaker}, bilinen başarılı konuşmacı doğrulama tekniklerinden biri olan birleşik etmen
analizi (joint factor analysis - JFA) ile konuşmacı doğrulama deneylerinde normal durumda 0.71\% eşit hata oranı (EER)
değeri elde edilirken, önceden kaydedilmiş (tekrar saldırıları) seslerin yanlış sınama sesleri ile yer değiştirilmesi
neticesinde, sistemin tekrar kayıtlarının \%68'ini kabul ettiğini ortaya koymuştur
\cite{villalba2010speaker}. Başarılı yöntemlerin tekrar saldırıları karşısında bu ölçüdeki savunmasızlığı,
yeni bir arayışı beraberinde getirmiştir. Bu, tekrar saldırılarının tespit edilebilmesi problemidir. Bu kapsamda, 2017
yılında \emph{otomatik konuşmacı doğrulama yanıltma ve saldırı tespiti (Automatic speaker verification spoofing and
    countermeasures challenge - ASVspoof 2017)} düzenlenmiştir \cite{kinnunen2017asvspoof}. Bu yarışma
neticesinde yapılan çalışmalardan birinde \cite{li2017study}, farklı öznitelikler ile Gauss Karışım Modeli
(Gaussian mixture model - GMM), destek vektör makineleri (support vector machines - SVM) ve derin sinir ağları (deep
neural networks - DNN) sınıflandırıcıları kullanılarak, tekrar saldırılarının tespit edilmesi problemi ele alınmış
olup; öznitelik çıkarmada yüksek frekans bölgesi analizinin ve DNN sınıflandırıcısının daha iyi sonuç verdiği
gösterilmiştir.
Yüksek frekans bölgesi analizinin, tekrar saldırılarının tespitinde performansı iyileştirdiğini gösteren bir diğer
çalışmada \cite{witkowski2017audio} ise; öznitelikler 6000-8000 Hz frekans aralığından çıkarıldığında \%3.38 EER değeri
elde edilmiştir. ASVspoof 2017 yarışmasında verilen referans saldırı tespiti yöntemine (GMM sınıflandırıcısı ve sabit
Q dönüşümü kepstral katsayıları) nazaran daha iyi performansın elde eden bir diğer çalışmada
\cite{nagarsheth2017replay} ise; derin öğrenme yöntemlerinin karşılaştırılması yapılmıştır. Tüm bu çalışmalar
göstermektedir ki; tekrar saldırıları KD sistemleri için önemli bir tehdittir ve DNN sınıflandırıcısı ile geliştirilen
sistemlerin performansı bu çalışmada motivasyon kaynağı olmuştur.

Bu çalışmada iki farklı yöntem kullanılarak, ses sinyalinden öznitelikler çıkarılmıştır. Bu öznitelikler kullanılarak;
ses sinyalinin gerçek veya sahte sınıfa ait olup olmadıklarının ayırt edilmesi için derin sinir ağları yöntemiyle bir
model oluşturulmuştur. Daha sonra bu modelden elde edilen olasılıklar, sistem performansını tespit etmek için; ses
işleme problemlerinde yaygın olarak kullanılan skor hesaplama yöntemine çevrilmiştir. DNN ile tekrar saldırısı tespit
sonuçları, ASVspoof 2017 organizasyon komitesi tarafından önerilen referans yöntem olan sabit Q transformu kepstrum
katsayılarının GMM sınıflandırıcısı ile modellenmesi (CQCC-GMM) yöntemi sonuçları ile karşılaştırılmıştır. Çalışmanın
devamı; tekrar saldırısı tespit yöntemi, tercih edilen öznitelik çıkarma yöntemleri, deneysel çalışmalar, deneysel
sonuçlar ve tartışma şeklinde devam etmektedir.

\section{Tekrar Saldırılarının Tesp{\footnotesize İ}t{\footnotesizeİ}}
Konuşmacı tanıma sistemlerine yapılabilecek tekrar saldırılarının tespiti aslında iki sınıflı bir örüntü tanıma
problemidir. Bu sınıflardan biri \emph{gerçek/orijinal} sınıf iken diğeri ise \emph{sahte} sınıftır. Tekrar
saldırılarının tespiti ise verilen bir ses sinyalinin orijinal veya daha önceden kaydedilmiş bir ses olup olmadığının
tespit edilmesi problemidir.

Saldırı tespiti, aslında iki sınıf arasında karar verme işlemi olup, karar için logaritmik olabilirlik oran
(log-likelihood ratio - $\mathcal{LLR}$) skoru kullanılabilir:
\begin{equation}
    \mathcal{LLR}=\log p(\mathbf{X}|C_1)-\log p(\mathbf{X}|C_2)
\end{equation}

Burada, $C_1$ ve $C_2$ sırası ile \emph{gerçek/orijinal} ve \emph{tekrar/sahte} sınıflarını temsil eden akustik
modellerdir ve her bir sınıfa ait eğitim öznitelikleri kullanılarak bu modeller eğitilebilir. Bu çalışmada kullanılan
ASVspoof 2017 veritabanı ile tekrar saldırısı tespiti için organizasyon komitesi tarafından Gauss Karışım Modeli (GMM)
sınıflandırıcısı referans modelleme yöntemi olarak önerilmiştir.

GMM yönteminde, her bir sınıf (orijinal ve tekrar), $M$ adet çok boyutlu Gauss yoğunluk fonksiyonunun
ağırlıklandırılmış toplamı şeklinde, $p(\mathbf{x}|\lambda)=\sum_{i=1}^Mw_ip_i(\mathbf{x}$), temsil edilir. Burada
$p_i(\mathbf{x})$ ortalaması $\mathbf{\mathrm{\mu_i}}$, ortak değişinti matrisi $\mathbf{\Sigma}_i$ olan $D-$boyutlu
($D$:öznitelik vektörlerinin boyutu) Gauss yoğunluk fonksiyonunu belirtmekte olup bütün bir GMM
$\lambda=\{w_i,\:\mathrm{\mathbf{\mu}}_i,\:\mathbf{\Sigma}_i\}_{i=1}^M$ şeklinde gösterilir. GMM yönteminin eğitim
aşamasında, her bir sınıfın eğitim öznitelik vektörleri
$\mathbf{X}=\{\mathbf{x}_1,\:\mathbf{x}_2,\:\ldots,\:\mathbf{x}_N \}$ eğitim öznitelikleri kullanılarak en büyük
olabilirlik kriteri kullanılarak beklentinin maksimumlaştırılması algoritması ile GMM parametreleri (ağırlık,
ortalama ve ortak değişinti) tahmin edilir. Test aşamasında ise; test ses sinyalinden elde edilen öznitelik
vektörleri, $\mathbf{Y}=\{\mathbf{y}_1,\:\ldots,\:\mathbf{y}_T\}$, kullanılarak logaritmik olabilirlik oranı skoru,

\begin{equation}
    \mathcal{LLR}=\log p(\mathbf{Y}|\lambda_\mathrm{gercek})-\log p(\mathbf{Y}|\lambda_\mathrm{tekrar})
\end{equation}
şeklinde hesaplanır. Burada $\lambda_\mathrm{gercek}$ ve $\lambda_\mathrm{tekrar}$ sırası ile gerçek ve tekrar
sınıflarına ait GMM'leri belirtmektedir. $p(\mathbf{Y}|\lambda)$ $\mathbf{Y}$ test öznitelik vektörleri için GMM
olabilirlik değeri olup şu şekilde hesaplanmaktadır:
\begin{equation}
    p(\mathbf{Y}|\lambda) = \prod_{t=1}^T \sum_{i=1}^M w_i p_i(\mathbf{y}_i|\lambda).
\end{equation}

Bu çalışmada, referans sınıflandırıcı yöntem olan Gauss karışım modeli tekniğine ek olarak derin sinir ağları
(deep neural network - DNN) modelleme yöntemi kullanılmıştır. DNN son yıllarda popüler olarak kullanılan bir yaklaşım
olup, farklı çalışmalarda hem sınıflandırma hem de öznitelik çıkarma amacı ile kullanılmıştır.

DNN sınıflandırıcısı ile gerçek ve tekrar seslerini birbirinden ayırt etmek için farklı öznitelikler ile ASVspoof 2017
veri tabanı eğitim kümesini kullanarak bir derin sinir ağı eğitilmiştir. Giriş katmanındaki nöron sayısı, kullanılan
özniteliklerin boyutu ile aynı olup, gizli katmanlarda farklı sayılarda nöron kullanılmıştır. Çıkış katmanında ise
biri gerçek, diğeri tekrar/sahte sınıfı temsil eden softmax aktivasyon fonksiyonunun kullanıldığı iki nöron
kullanılmıştır. Her bir çıkış nöronu, ilgili sınıfın sonsal (posterior) olasılığını temsil ettiğinden, sonsal
olasılıklar, logaritmik olabilirlik oranı skoruna şu şekilde dönüştürülmüştür:

\begin{equation}
    \mathcal{LLR}=\log p(C_1|\mathbf{X})-\log p(C_2|\mathbf{X})
\end{equation}

\section{{\ Ö}zn{\footnotesizeİ}tel{\footnotesizeİ}k {\ Ç}ıkarma}
DNN yönteminin güçlü yönlerinden biri sınıflandırma için gerekli olan ve ayırt ediciliği yüksek olan öznitelikleri
kendisinin çıkarması olsa da; ses sinyallerinin kullanıldığı çalışmaların çoğunda ön-işleme adımında çıkarılan
öznitelikler DNN algoritmasına giriş olarak verilmektedir \cite{nagarsheth2017replay}. Bu çalışmada iki tür öznitelik
çıkarım yöntemi tercih edilmiştir:

\subsection{Mel-Frekansı Kepstrum Katsayıları}
Mel-frekansı kepstrum katsayıları (MFCC) \cite{kinnunen2010overview,davis1980comparison}, konuşma analizi
problemlerinde en yaygın kullanılan öznitelik çıkarma yöntemlerinden biridir. MFCC öznitelikleri, ön vurgulama yapılan
ses sinyalinin her biri 25 ms uzunluğunda, 10 ms'lik kısımları birbiri ile örtüşen Hamming pencere ile pencerelenmiş
çerçevelerden elde edilir. Pencerelenmiş çerçevelerin ayrık Fourier dönüşümü (DFT) alınarak genlik spektrumları elde
edilir ve üçgen süzgeçlerden oluşan süzgeç takımından geçirilir. Logaritmik süzgeç çıkışlarının ayrık kosinüs
dönüşümünün alınmasıyla MFCC öznitelikeri elde edilir. Burada öznitelik vektörünün ilk katsayısı logaritmik enerjidir.
Bu yöntem için açık kaynak kodlu dil ve konuşmacı tanıma paketi SIDEKIT \cite{sidekit} kullanılmıştır.

\subsection{Uzun Dönem Ortalama Spektrum İstatistikleri}
Uzun dönem ortalama spektrum (LTAS) istatistikleri \cite{muckenhirn2016presentation}  özniteliklerini elde etmek için;
$s$ ses sinyali öncelikle ön-vurgulama süzgecinden geçirilerek birbiri ile örtüşen kısa çerçevelere bölünür. Hamming
pencere fonksiyonu ile pencerelenen her bir çerçevenin $N$ noktalı ayrık Fourier dönüşümü (AFD) alınarak genlik
spektrumu hesaplanır. $S_t(k)$, $t=1,2,\:\ldots,T$, t. çerçevenin Fourier dönüşümünü belirtmek üzere
($k=1,\:\ldots,\:N/2)$ uzun dönem ortalama ve varyans istatistikleri elde edilir:

\begin{equation}
    \mu(k) = \frac{1}{T}\sum_{t=1}^{T} \log \left|S_t(k)\right|
\end{equation}

\begin{equation}
    \sigma^2(k) = \frac{1}{T}\sum_{t=1}^{T}(\log \left|S_t(k)\right| - \mu(k))^2
\end{equation}

Son olarak, ortalama ve varyans istatistikleri birleştirilerek, uzun dönem spektral istatistikler öznitelik vektörleri
elde edilir.

\section{Deneysel {\ Ç}alışmalar}
Deneyler sırasında ASVspoof 2017\cite{kinnunen2017reddots} veritabanı kullanılmıştır. ASVspoof 2017 veritabanı 16
kHz'de örneklenmiş, 16 bit çözünürlüğünde kaydedilmiş seslerden oluşmaktadır. Bu sesler, Tablo~\ref{Dataset}'de
gösterildiği gibi eğitim, geliştirme ve değerlendirme olmak üzere birbiri ile örtüşmeyen üç alt kümeye ayrılmıştır.
Eğitim verisi; gerçek ve sahte sınıflara ait akustik modellerin eğitilmesi amacı ile, geliştirme kümesi, kullanılan
sistemin parametrelerini optimize ederek en iyi saldırı tespiti sistemini oluşturmak, değerlendirme kümesi ise en iyi
modelin performansının genelleştirilebilirliği amacıyla kullanılmıştır. Veritabanı ile ilgili daha kapsamlı bilgiye
\cite{kinnunen2017asvspoof} çalışmasından ulaşılabilir.

\begin{table}[ph]
    \centering
    \caption{\textsc{ASVspoof 2017 ver{\footnotesize  İ} tabanı}}
    \label{Dataset}
    \begin{tabular}{|c|c|cc|}
        \hline
        Alt Küme      & Konuşmacı Sayısı & \multicolumn{2}{c|}{Kayıt Sayısı}            \\
                      &                  & Gerçek                            & Sentetik \\ \hline
        Eğitim        & 10               & 1507                              & 1507     \\
        Geliştirme    & 8                & 760                               & 950      \\
        Değerlendirme & 24               & 1294                              & 11987    \\
        \hline
    \end{tabular}
\end{table}

Sınıfladırıcı olarak Bölüm II'de bahsedilen DNN yöntemi kullanılmıştır. MFCC öznitelikleri için üç ve uzun dönem
ortalama spektrum istatistikleri (LTAS) öznitelikleri için ise beş adet gizli katmandan oluşan DNN kullanılmıştır.
Hata fonksiyonunu optimize etmek için Stochastic Gradient Descent (SGD) algoritması kullanılmıştır. Her iki öznitelik
türü kullanılarak farklı sayılarda nöron ve gizli katman içeren DNN sınınflandırcısı ile yapılan deneylerle ideal
parametreler tespit edimiştir. MFCC ve LTAS öznitelik vektörlerinin boyutları ve karakteristikleri benzer olmadığından,
öznitelikler için tespit edilen ideal DNN yapısı farklıdır. Burada MFCC için giriş katmanı nöron sayısı; kullanılan
öznitelik sayısı (19 adet) ve ilave olarak logaritmik enerji katsayısı olmak üzere 20, her bir gizli katman ise 1024
birimden oluşmaktadır. LTAS öznitelikleri için ise 256 adet ortalama, 256 standart sapma istatistikleri birleştirilerek
tek bir öznitelik vektörü oluşturulduğundan giriş katmanında 512, gizli katmanlarda ise 1024 nöron bulunmaktadır. Her
iki durum için de, gizli katmanlarda ReLU aktivasyon fonksiyonu kullanılmıştır. KD sistemler için tekrar saldırı
tespiti ikili sınıflandırma problemi olduğundan; çıkış katmanlarının nöron sayısı iki ve aktivasyon fonksiyonu softmax
olarak belirlenmiştir. İki adet çıkış birimi sırası ile gerçek ve sahte sınıflara karşılık gelmektedir. Bu deneyler;
açık kaynak kodlu derin öğrenme kütüphanesi Keras\cite{keras} kullanılarak gerçekleştirilmiştir.

Aşırı eğitimi (over training) engellemek amacıyla, öğrenme oranı (learning rate) düşük seçilmiş ve her bir katmandan
sonra 0.75 değerinde dropout\cite{srivastava2014dropout} eklenmiştir. Öğrenme oranının düşük olması nedeniyle en iyi
modele ulaşmak için epoch (devir) sayısı yüksek (10000) seçilmiştir. Diğer parametrelerin sisteme etkisini azaltmak
\cite{li2014efficient} ve hafıza problemlerini aşmak için; eğitim verisi gruplara ayrıştırılarak eğitilmiştir (batch
training). Eğitilen modelin, geliştirme verisi kullanılarak belirli bir frekansta validasyonu yapılmış ve validasyon
skoru bir önceki sonuçtan iyi olduğu tespit edilen model en iyi model olarak kaydedilmiştir. Eğitim zamanından tasarruf
amacı ile, validasyon skoru azalmadığında veya artmaya başladığında 10000 epoch sayısı beklenmeden eğitim işlemi
sonlandırılmıştır (early stopping).

ASVspoof 2017 yarışmasında tekrar saldırılarının tespiti problemi için performans metriği olarak eşit hata oranı
(equal error rate - EER) kullanılmıştır \cite{kinnunen2017asvspoof,kinnunen2017asvspoof1}. EER değeri, yanlış kabul ve
yanlış ret oranlarının birbirine eşit olduğu eşik değerdeki hata oranına karşılık gelmektedir. Ayrıca EER metriğine
ilaveten, önerilen sistemin bütün performansını görüntülemek amacıyla sezim hata ödünleşimi (detection error
trade-off - DET) \cite{martin1997det} eğrileri de verilmiştir.

\section{Deneysel Sonuçlar}
Deneyler sonucunda MFCC ve LTAS öznitelikleri ile geliştirme ve değerlendirme kümesi için elde edilen sonuçlar
Tablo~\ref{EHO}'de verilmiştir. LTAS öznitelikleri ile geliştirme kümesinde $\%4.55$, değerlendirme kümesinde ise
$\%18.10$ EER değerleri elde edilmiştir. MFCC öznitelikleri için ise sırasıyla $\%18.78$ ve $\%24.81$ EER değerleri
elde edilmiştir. Sonuçlara göre LTAS özniteliklerinin bu sınıflandırıcı yapısı için, MFCC özniteliklerine göre daha
iyi bir seçim olduğu açıkça görülmektedir. Sistem performansının daha iyi gözlemlenebilmesi amacıyla; ASVspoof 2017
yarışmasının temel sistemi olan CQCC-GMM \cite{kinnunen2017asvspoof,kinnunen2017asvspoof1} sonuçları da tabloda
verilmektedir. Görüldüğü gibi, geliştirme kümesinde LTAS öznitelikleri ve DNN sınıflandırıcısı referans sistemden
yaklaşık $2.5$ kat daha iyi başarım göstermektedir. Fakat MFCC öznitelikleri DNN sınıflandırıcısı ile birlikte
kullanıldığında geliştirme kümesinde referans sisteme nazaran daha yüksek hata oranı elde edilmiştir.

\begin{table}[ph]
    \centering
    \caption{\textsc{ASVspoof 2017 ver{\footnotesize  İ} tabanı {\footnotesize  İ}le elde ed{\footnotesize  İ}len EER
    de{\footnotesize  Ğ}erler{\footnotesize  İ}}}
    \label{EHO}
    \begin{tabular}{|c|c|cc|c|}
        \hline
        Öznitelik & Sistem & \multicolumn{2}{c|}{Alt Küme}                 \\
                  &        & Geliştirme                    & Değerlendirme \\
        \hline
        MFCC      & DNN    & 18.78                         & 24.81         \\
        LTAS      & DNN    & 4.55                          & 18.10         \\
        CQCC      & GMM    & 11.85                         & 30.00         \\
        \hline
    \end{tabular}
\end{table}

Değerlendirme kümesi sonuçları göz önüne alındığında, önerilen sistemin (LTAS öznitelikleri ve DNN sınıflandırıcısı)
referans sisteme (GMM sınıflandırıcısı ve CQCC öznitelikleri) göre daha yüksek performans gösterdiği görülmektedir.
Benzer şekilde MFCC öznitelikleri ile de referans sistemden daha iyi performans elde edilmiştir.

Son olarak Şekil~\ref{DetPLot} ve Şekil~\ref{DetPLotEval}'de her iki öznitelik türü (MFCC ve LTAS) ve referans sistem
(GMM-CQCC) ile sırası ile geliştirme ve değerlendirme alt kümeleri için elde edilen DET eğrileri verilmiştir.

\begin{figure}[htbp]
    \centering
    \shorthandoff{=}
    \includegraphics[scale=0.24]{./Images/development-det-curve.png}
    \shorthandon{=}
    \caption{Geliştirme kümesi için elde edilen DET grafiği.}
    \label{DetPLot}
\end{figure}
\begin{figure}[htbp]
    \centering
    \shorthandoff{=}
    \includegraphics[scale=0.22]{./Images/evaluation-det-curve.png}
    \shorthandon{=}
    \caption{Değerlendirme kümesi için elde edilen DET grafiği.}
    \label{DetPLotEval}
\end{figure}

\section{Tartışma}
Bu çalışmada; daha önceden kaydedilen bir sesin tekrar oynatılması yöntemiyle, KD sistemlerine yapılması olası
saldırıların tespit edilmesi problemi ele alınmıştır. MFCC ve uzun dönem ortalama spektrum (LTAS) istatistikleri
yöntemleriyle elde edilen öznitelikler, DNN kullanılarak sınıflandırılmıştır. MFCC özniteliklerinin vektör boyutunun
çeşitli yöntemlerle büyütülerek kullanılması DNN sistemler için daha iyi sonuç verebileceği düşünülmektedir. Ayrıca
DNN yapısının ASVspoof 2017 yarışmasında belirlenen referans sisteme (CQCC - GMM) nazaran daha iyi olduğu
gözlemlenmiştir. Farklı öznitelik yöntemleri de kullanılarak daha iyi bir DNN yapısı ile iyi sonuçlar alınabileceği
gelecek çalışmalar için motivasyon kaynağıdır.

\section*{Teşekkür}
Bu çalışma TÜBİTAK (proje numarası 115E916) tarafından desteklenmiştir.

\bibliographystyle{IEEEtran}
\bibliography{bibliography}

\end{document}
